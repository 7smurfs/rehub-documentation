\chapter{Zaključak i budući rad}
		
		Suočili smo se s izazovom razvoja web aplikacije za naručivanje medicinske rehabilitacije. Vrijednost koju smo najviše cijenili bila je suradnja u timu, koja je donijela raznolike izazove s kojima smo se uspješno nosili. Voditelj grupe imao je zahtjevan zadatak koordinacije sedmero ljudi, što nije bilo jednostavno. Početni susret s mnogim novim konceptima zahtijevao je brzo učenje i intenzivan rad na aplikaciji i dokumentaciji.

        Razumijevanje alata i tehnologija korištenih u izradi aplikacije i dokumentacije predstavljalo je dodatan izazov koji je zahtijevao značajan vremenski angažman. S vremenom proces je postao olakšan jer smo se bolje prilagodili timskom radu i aktivno si pružali podršku. Iako smo uvjereni da bi aplikacija mogla ostvariti još veći napredak s više vremena na raspolaganju, zadovoljni smo končanim rezultatom s obzirom na fakultetske i druge obaveze svakog člana tima.

        Prijedlozi za budući rad obuhvaćaju proširenje funkcionalnosti aplikacije, kao što su dodatne opcije za korisnike, poboljšane analize statistike rehabilitacije te implementacija dodatnih sigurnosnih slojeva. Također, istraživanje mogućnosti integracije novih tehnologija i prilagodba sučelja prema povratnim informacijama korisnika predstavljaju ključne smjernice za daljnji razvoj "ReHub" aplikacije.

        Kvalitetna komunikacija među članovima tima bila je ključna. Koristili smo alate poput WhatsAppa, Teamsa i Atlassian platforme. Redovito održavani sastanci služili su za raspravu o zadacima, pregled dosadašnjeg rada te dodjeljivanje novih zadataka. Sastanci su se održavali prema potrebi. 
        
        Na kraju, prepoznali smo da su opuštena atmosfera, međusobna pomoć i organizirano vođenje od strane voditelja ključni faktori za postizanje krajnjeg cilja.
        		
		\eject 