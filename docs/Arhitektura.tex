\chapter{Arhitektura i dizajn sustava}

Rehub predstavlja sofisticirano arhitektonsko rješenje koje integrira PostgreSQL bazu podataka, Spring Boot za backend logiku, React za korisničko sučelje, te Tailwind CSS za moderno i responzivno oblikovanje. Sustav je implementiran na cloud platformama Render za backend i bazu te Netlify za frontend, koristeći njihove usluge za brz, skalabilan i pouzdan deploying.\\

\begin{packed_item}
	\large \item Baza podataka: PostgreSQL \normalsize \\
	PostgreSQL, snažan sustav za upravljanje relacijskim bazama podataka, predstavlja ključnu komponentu u arhitekturi ove web aplikacije. Struktura baze podataka temeljito je dizajnirana kako bi zadovoljila specifične zahtjeve aplikacije. Kroz pažljivu integraciju s PostgreSQL-om, aplikacija ostvaruje brz i učinkovit pristup podacima, čime se osigurava optimalna performansa. Ova konfiguracija baze podataka predstavlja ključnu komponentu cjelokupnog arhitektonskog rješenja, pružajući čvrstu temeljnu strukturu za rad backend sustava. \\
	\large \item Backend: Spring Boot \normalsize \\
	Spring Boot, kao Java-based framework, pruža snažan temelj za razvoj backend logike. Ova web aplikacija koristi Spring Boot za stvaranje RESTful API-ja koji komunicira s PostgreSQL bazom podataka. Ovo rješenje omogućuje efikasno upravljanje podacima i pruža mogućnost proširivosti sustava kroz modularnost i lakoću integracije. \\
	\large \item Frontend: React \normalsize \\
	React, centralna JavaScript biblioteka, određuje arhitekturu frontend dijela ove aplikacije. Modularne React komponente omogućuju čist i jednostavan kod, prilagodljiv specifičnim zahtjevima sučelja. Ovaj pristup omogućuje stvaranje brzog, dinamičkog korisničkog sučelja koje se lako održava. Kroz React, osigurava se fluidna navigacija kroz aplikaciju, pridonoseći ukupnom intuitivnom korisničkom iskustvu. \\
	\large \item Frontend Integracija: Tailwind CSS, Axios, React-Router-Dom, reCAPTCHA \normalsize \\
	Frontend aplikacije unaprijeđen je integracijom Tailwind CSS za brzo oblikovanje sučelja, Axios za efikasnu komunikaciju s backendom, React-Router-Dom za fluidnu navigaciju te reCAPTCHA sustava radi dodatne sigurnosti. Tailwind CSS pridonosi estetskom izgledu, Axios omogućuje dinamičku interakciju s backendom, React-Router-Dom olakšava upravljanje rutama, dok reCAPTCHA sprječava zlonamjerne aktivnosti. Kroz ovu integraciju postiže se uravnoteženo frontend rješenje koje kombinira funkcionalnost, estetiku i sigurnost korisničkog iskustva. \\
	\large \item Deployment: Render za backend i Netlify za frontend \normalsize \\
	Backend i baza podataka su postavljeni na Render platformu za optimalno upravljanje resursima. Render osigurava automatsko skaliranje i brzu dostupnost aplikacije koristeći njihove cloud usluge. Na renderu je također podešeno kontinuirano postavljanje koje omogućuje automatsko ažuriranje softvera. Frontend je zasebno postavljen na Netlify platformi, koja pruža brzo i globalno dostupno CDN (Content Delivery Network). \\
\end{packed_item}

Rehub predstavlja integrirano arhitektonsko rješenje koje kombinira najbolje prakse u razvoju aplikacija, pružajući stabilnost, skalabilnost i visoku razinu korisničkog iskustva. Korištenje navedenih tehnologija i servisa osigurava da aplikacija bude spremna za izazove suvremenog web razvoja.




\begin{figure}[H]
	\includegraphics[scale=0.25]{dijagrami/architecture_diagram.png}
	\centering
	\caption{Dijagram arhitekture}
	\label{fig:architecture_diagram}
\end{figure}

\section{Baza podataka}

U bazi su stvorene relacije koje olakšavaju jednostavno rukovanje podacima, uključujući dodavanje, brisanje, izmjenu i dohvaćanje podataka. Svi entiteti i relacije su organizirani prema trećoj normalnoj formi kako bi se izbjegla redundancija podataka. Za bolje vizualiziranje strukture baze podataka, napravljen je relacijski dijagram koji je prikazan u nastavku. \\
\\
Baza podataka se sastoji od tablica (relacija) koje su definirane svojim imenom i skupom
atributa. Baza podataka ove aplikacije sastoji se od sljedećih entiteta:

\begin{packed_item}
	\item rehub\_user
	\item employee
	\item user\_role
	\item role
	\item patient
	\item therapy
	\item therapy\_result
	\item doctor
	\item appointment
	\item room
	\item equipment
	\item faq
	\item personal\_data
	\item verification
	\item password\_reset
	\item flyway\_history\_schema
\end{packed_item}


\subsection{Opis tablica}

Prva ćelija svake tablice označava njeno ime. U prvom stupcu navedeni su atributi tog entiteta, u drugom stupcu naveden je tip varijable, a u trećem opis svakog pojedinog atributa. Svjetlozelenom bojom označeni su primarni ključevi. Svjetlo plavom označeni su strani ključevi. \\	


\begin{longtblr}[
	label=none,
	entry=none,
	]{
		width=\textwidth,
		colspec={|X[6,l]|X[6,l]|X[20,l]|}, 
		rowhead=1,
	}
	\hline
	\SetCell[c=3]{c}{\textbf{rehub\_user}} \\ \hline[3pt]
	\SetCell{LightGreen} ID & BIGSERIAL & JEDINSTVENI IDENTIFIKATOR KORISNIKA \\ \hline
	USERNAME & VARCHAR & KORISNIČKO IME S KOJIM SE KORISNIK PRIJAVLJUJE \\ \hline
	PASSWORD & VARCHAR & LOZINKA S KOJOM SE KORISNIK PRIJAVLJUJE \\ \hline
	STATUS & VARCHAR & STATUS KORISNIKA \\ \hline
	\SetCell{LightBlue} PERSONAL\_DATA\_ID & VARCHAR & JEDINSTVENI IDENTIFIKATOR KORISNIKA \\ \hline
\end{longtblr}

Entitet označava korisnika aplikacije te sadrži atribute: id, username, password, status, personal\_data\_id.

% Table 2: employee
\begin{longtblr}[
	label=none,
	entry=none,
	]{
		width=\textwidth,
		colspec={|X[6,l]|X[6,l]|X[20,l]|}, 
		rowhead=1,
	}
	\hline
	\SetCell[c=3]{c}{\textbf{employee}} \\ \hline[3pt]
	\SetCell{LightGreen} ID & BIGSERIAL & JEDINSTVENI IDENTIFIKATOR ZAPOSLENIKA \\ \hline
	FIRST\_NAME & VARCHAR & IME ZAPOSLENIKA \\ \hline
	LAST\_NAME & VARCHAR & PREZIME ZAPOSLENIKA \\ \hline
	PHONE\_NUMBER & VARCHAR & BROJ MOBITELA ZAPOSLENIKA \\ \hline
	PROFESSION & VARCHAR & STRUKA ZAPOSLENIKA \\ \hline
	GENDER & VARCHAR & SPOL ZAPOSLENIKA \\ \hline
	CREATED\_AT & TIMESTAMP & TRENUTAK KREIRANJA \\ \hline
	LAST\_MODIFIED\_AT & TIMESTAMP & TRENUTAK ZADNJE PROMJENE \\ \hline
	\SetCell{LightBlue} USER\_ID & VARCHAR & JEDINSTVENI IDENTIFIKATOR KORISNIKA \\ \hline
\end{longtblr}

Entitet označava osobe zaposlene u klinici. Atributi ovoga entiteta su: id, first\_name, last\_name, phone\_number, profession, gender, created\_at, last\_modified\_at, user\_id.

% Table 3: user_role
\begin{longtblr}[
	label=none,
	entry=none,
	]{
		width=\textwidth,
		colspec={|X[6,l]|X[6,l]|X[20,l]|}, 
		rowhead=1,
	}
	\hline
	\SetCell[c=3]{c}{\textbf{user\_role}} \\ \hline[3pt]
	\SetCell{LightBlue} USER\_ID & BIGSERIAL & JEDINSTVENI IDENTIFIKATOR KORISNIKA \\ \hline
	\SetCell{LightBlue} ROLE\_ID & BIGSERIAL & JEDINSTVENI IDENTIFIKATOR VRSTE KORISNIKA \\ \hline
\end{longtblr}

Entitet označava vrstu korisnika aplikacije, a njegovi atributi su: user\_id, role\_id.

% Table 4: role
\begin{longtblr}[
	label=none,
	entry=none,
	]{
		width=\textwidth,
		colspec={|X[6,l]|X[6,l]|X[20,l]|}, 
		rowhead=1,
	}
	\hline
	\SetCell[c=3]{c}{\textbf{role}} \\ \hline[3pt]
	\SetCell{LightGreen} ID & BIGSERIAL & JEDINSTVENI IDENTIFIKATOR ULOGE \\ \hline
	NAME & VARCHAR & IME ULOGE \\ \hline
\end{longtblr}

Entitet označava ulogu i sastoji se od atributa: id, name.

% Table 5: patient
\begin{longtblr}[
	label=none,
	entry=none,
	]{
		width=\textwidth,
		colspec={|X[6,l]|X[6,l]|X[20,l]|}, 
		rowhead=1,
	}
	\hline
	\SetCell[c=3]{c}{\textbf{patient}} \\ \hline[3pt]
	\SetCell{LightGreen} ID & BIGSERIAL & JEDINSTVENI IDENTIFIKATOR PACIJENTA \\ \hline
	FIRST\_NAME & VARCHAR & IME PACIJENTA \\ \hline
	LAST\_NAME & VARCHAR & PREZIME PACIJENTA \\ \hline
	GENDER & VARCHAR & SPOL PACIJENTA \\ \hline
	PHONE\_NUMBER & VARCHAR & BROJ MOBITELA PACIJENTA \\ \hline
	DATE\_OF\_BIRTH & DATE & DATUM ROĐENJA PACIJENTA \\ \hline
	CREATED\_AT & TIMESTAMP & TRENUTAK KREIRANJA \\ \hline
	LAST\_MODIFIED\_AT & TIMESTAMP & TRENUTAK ZADNJE PROMJENE \\ \hline
	\SetCell{LightBlue} USER\_ID & BIGSERIAL & JEDINSTVENI IDENTIFIKATOR KORISNIKA \\ \hline
\end{longtblr}

Entitet predstavlja pacijenta koji je korisnik aplikacije. Atributi su: id, first\_name, last\_name, gender, phone\_number, date\_of\_birth, created\_at, last\_modified\_at, user\_id.

% Table 6: therapy
\begin{longtblr}[
	label=none,
	entry=none,
	]{
		width=\textwidth,
		colspec={|X[6,l]|X[6,l]|X[20,l]|}, 
		rowhead=1,
	}
	\hline
	\SetCell[c=3]{c}{\textbf{therapy}} \\ \hline[3pt]
	\SetCell{LightGreen} ID & BIGSERIAL & JEDINSTVENI IDENTIFIKATOR TERAPIJE \\ \hline
	TYPE & VARCHAR & VRSTA TERAPIJE \\ \hline
	REQUEST & VARCHAR & ZAHTJEV ZA TERAPIJOM \\ \hline
	STATUS & VARCHAR & STATUS TERAPIJE \\ \hline
	REF\_ID & BIGSERIAL & JEDINSTVENI IDENTIFIKATOR \\ \hline
	CREATED\_AT & TIMESTAMP & TRENUTAK KREIRANJA \\ \hline
	LAST\_MODIFIED\_AT & TIMESTAMP & TRENUTAK ZADNJE PROMJENE \\ \hline
	\SetCell{LightBlue} PATIENT\_ID & BIGSERIAL & JEDINSTVENI IDENTIFIKATOR PACIJENTA \\ \hline
	\SetCell{LightBlue} EMPLOYEE\_ID & BIGSERIAL & JEDINSTVENI IDENTIFIKATOR ZAPOSLENIKA \\ \hline
	\SetCell{LightBlue} ROOM\_ID & BIGSERIAL & JEDINSTVENI IDENTIFIKATOR SOBE \\ \hline
	\SetCell{LightBlue} APPOINTMENT\_ID & BIGSERIAL & JEDINSTVENI IDENTIFIKATOR TERMINA \\ \hline
	\SetCell{LightBlue} THERAPY\_RESULT\_ID & BIGSERIAL & JEDINSTVENI IDENTIFIKATOR REZULTATA TERAPIJE \\ \hline
	DOCTOR\_FULL\_NAME & VARCHAR & PUNO IME DOKTORA \\ \hline
	THERAPY\_SCAN & VARCHAR & NAZIV UPUTNICE ZA TERAPIJU \\ \hline
\end{longtblr}

Entitet opisuje terapiju dodijeljenu pacijentu. Njeni atributi su: id, type, request, status, ref\_id, created\_at, last\_modified\_at, patient\_id, employee\_id, room\_id, appointment\_id, therapy\_result\_id, doctor\_full\_name, therapy\_scan.

% Table 7: doctor
\begin{longtblr}[
	label=none,
	entry=none,
	]{
		width=\textwidth,
		colspec={|X[6,l]|X[6,l]|X[20,l]|}, 
		rowhead=1,
	}
	\hline
	\SetCell[c=3]{c}{\textbf{doctor}} \\ \hline[3pt]
	\SetCell{LightGreen}ID & BIGSERIAL & JEDINSTVENI IDENTIFIKATOR DOKTORA \\ \hline
	FIRST\_NAME & VARCHAR & IME DOKTORA \\ \hline
	LAST\_NAME & VARCHAR & PREZIME DOKTORA \\ \hline
	PIN & VARCHAR & PIN DOKTORA \\ \hline
	MEDICAL\_NO & VARCHAR & MEDICINSKI BROJ \\ \hline
	SPECIALITY & VARCHAR & DOKTOROVA SPECIJALIZACIJA \\ \hline
	DATE\_OF\_EMPLOYMENT & DATE & DATUM ZAPOSLENJA \\ \hline
	DATE\_OF\_REGISTRATION & DATE & DATUM REGISTRACIJE \\ \hline
	DATE\_OF\_BIRTH & TIMESTAMP & DATUM ROĐENJA \\ \hline
\end{longtblr}

Entitet predstavlja doktora koji je korisnik aplikacije. Atributi su: id, first\_name, last\_name, pin, medical\_no, speciality, date\_of\_employment, date\_of\_registration, date\_of\_birth.

% Table 8: appointment
\begin{longtblr}[
	label=none,
	entry=none,
	]{
		width=\textwidth,
		colspec={|X[6,l]|X[6,l]|X[20,l]|}, 
		rowhead=1,
	}
	\hline
	\SetCell[c=3]{c}{\textbf{appointment}} \\ \hline[3pt]
	\SetCell{LightGreen}ID & BIGSERIAL & JEDINSTVENI IDENTIFIKATOR TERMINA \\ \hline
	START\_AT & TIMESTAMP & POČETAK TERMINA \\ \hline
	END\_AT & TIMESTAMP & KRAJ TERMINA \\ \hline
\end{longtblr}

Entitet predstavlja termin na koji se pacijent naručio, a atributi su: id, start\_at, end\_at.

% Table 9: therapy_result
\begin{longtblr}[
	label=none,
	entry=none,
	]{
		width=\textwidth,
		colspec={|X[6,l]|X[6,l]|X[20,l]|}, 
		rowhead=1,
	}
	\hline
	\SetCell[c=3]{c}{\textbf{therapy\_result}} \\ \hline[3pt]
	\SetCell{LightGreen}ID & BIGSERIAL & JEDINSTVENI IDENTIFIKATOR REZULTATA TERAPIJE \\ \hline
	STATUS & VARCHAR & STATUS TERAPIJE \\ \hline
	RESULT & VARCHAR & KONAČNI REZULTAT TERAPIJE \\ \hline
\end{longtblr}

Ovaj entitet predstavlja rezultat terapije, a definiran je sljedećim atributima: id, status, result.

% Table 10: room
\begin{longtblr}[
	label=none,
	entry=none,
	]{
		width=\textwidth,
		colspec={|X[6,l]|X[6,l]|X[20,l]|}, 
		rowhead=1,
	}
	\hline
	\SetCell[c=3]{c}{\textbf{room}} \\ \hline[3pt]
	\SetCell{LightGreen}ID & BIGSERIAL & JEDINSTVENI IDENTIFIKATOR SOBE \\ \hline
	LABEL & VARCHAR & OZNAKA SOBE \\ \hline
	CAPACITY & BIGSERIAL & KAPACITET SOBE \\ \hline
	STATUS & VARCHAR & STATUS O KORIŠTENJU SOBE \\ \hline
	SPECIAL\_MESSAGE & VARCHAR & NAPOMENA \\ \hline
\end{longtblr}

Entitet predstavlja sobu u kojoj se odvija terapija, a njeni atributi su: id, label, capacity, status, special\_message.

% Table 11: equipment
\begin{longtblr}[
	label=none,
	entry=none,
	]{
		width=\textwidth,
		colspec={|X[6,l]|X[6,l]|X[20,l]|}, 
		rowhead=1,
	}
	\hline
	\SetCell[c=3]{c}{\textbf{equipment}} \\ \hline[3pt]
	\SetCell{LightGreen}ID & BIGSERIAL & JEDINSTVENI IDENTIFIKATOR OPREME \\ \hline
	NAME & VARCHAR & IME OPREME \\ \hline
	STATUS & VARCHAR & STATUS O KORIŠTENJU OPREME \\ \hline
	SPECIAL\_MESSAGE & VARCHAR & NAPOMENA \\ \hline
	\SetCell{LightBlue}ROOM\_ID & BIGSERIAL & JEDINSTVENI IDENTIFIKATOR SOBE \\ \hline
\end{longtblr}

Ovaj entiet se odnosi na opremu koja se koristi prilikom terapije, a opisuju ju: id, name, status, special\_message, room\_id.

% Table 12: personal_data
\begin{longtblr}[
	label=none,
	entry=none,
	]{
		width=\textwidth,
		colspec={|X[6,l]|X[6,l]|X[20,l]|}, 
		rowhead=1,
	}
	\hline
	\SetCell[c=3]{c}{\textbf{personal\_data}} \\ \hline[3pt]
	\SetCell{LightGreen}ID & BIGSERIAL & JEDINSTVENI IDENTIFIKATOR KORISNIKA \\ \hline
	FIRST\_NAME & VARCHAR & IME KORISNIKA \\ \hline
	LAST\_NAME & VARCHAR & PREZIME KORISNIKA \\ \hline
	PIN & VARCHAR & OIB KORISNIKA \\ \hline
	PHIN & VARCHAR & MATIČNI BROJ KORISNIKA \\ \hline
	DATE\_OF\_BIRTH & DATE & DATUM ROĐENJA KORISNIKA \\ \hline
\end{longtblr}

Ovaj entitet sadrži osobne podatke o korisniku aplikacije, a atributi su mu: id, first\_name, last\_name, pin, phin i date\_of\_birth.

% Table 13: faq
\begin{longtblr}[
	label=none,
	entry=none,
	]{
		width=\textwidth,
		colspec={|X[6,l]|X[6,l]|X[20,l]|}, 
		rowhead=1,
	}
	\hline
	\SetCell[c=3]{c}{\textbf{faq}} \\ \hline[3pt]
	\SetCell{LightGreen}ID & BIGSERIAL & JEDINSTVENI IDENTIFIKATOR PITANJA I ODGOVORA \\ \hline
	QUESTION & VARCHAR & PITANJE \\ \hline
	ANSWER & VARCHAR & ODGOVOR \\ \hline
\end{longtblr}

Entitet se odnosi na često postavljena pitanja i sastoji se od sljedećih atributa: id, question te answer.

% Table 14: verification
\begin{longtblr}[
	label=none,
	entry=none,
	]{
		width=\textwidth,
		colspec={|X[6,l]|X[6,l]|X[20,l]|}, 
		rowhead=1,
	}
	\hline
	\SetCell[c=3]{c}{\textbf{verification}} \\ \hline[3pt]
	\SetCell{LightGreen}ID & BIGSERIAL & JEDINSTVENI IDENTIFIKATOR VERIFIKACIJE \\ \hline
	TOKEN & VARCHAR & TOKEN VERIFIKACIJE \\ \hline
	STATUS & VARCHAR & STATUS VERIFIKACIJE \\ \hline
	\SetCell{LightBlue}USER\_ID & BIGSERIAL & JEDINSTVENI IDENTIFIKATOR KORISNIKA \\ \hline
\end{longtblr}

Entitet se odnosi na verifikaciju i sastoji se od sljedećih atributa: id, token, status i user\_id.

% Table 15: password_reset
\begin{longtblr}[
	label=none,
	entry=none,
	]{
		width=\textwidth,
		colspec={|X[6,l]|X[6,l]|X[20,l]|}, 
		rowhead=1,
	}
	\hline
	\SetCell[c=3]{c}{\textbf{password\_reset}} \\ \hline[3pt]
	\SetCell{LightGreen}ID & BIGSERIAL & JEDINSTVENI IDENTIFIKATOR PONOVNOG POSTAVLJANJA LOZINKE \\ \hline
	TOKEN & VARCHAR & TOKEN PONOVNOG POSTAVLJANJA LOZINKE \\ \hline
	STATUS & VARCHAR & STATUS PONOVNOG POSTAVLJANJA LOZINKE \\ \hline
	\SetCell{LightBlue}USER\_ID & BIGSERIAL & JEDINSTVENI IDENTIFIKATOR KORISNIKA \\ \hline
	CREATED\_AT & TIMESTAMP & TRENUTAK KREIRANJA \\ \hline
	LAST\_MODIFIED\_AT & TIMESTAMP & ZADNJE MIJENJANO \\ \hline
\end{longtblr}

Entitet se odnosi na ponovno postavljanje lozinke i sastoji se od sljedećih atributa: id, token, status, user\_id, created\_at i last\_modified\_at.





\subsection{Dijagram baze podataka}

\begin{figure}[H]
	\includegraphics[scale=0.35]{dijagrami/rehub_db_diagram.png}
	\centering
	\caption{Dijagram baze podataka}
	\label{fig:dbDiagram}
\end{figure}

\eject


\section{Dijagram razreda}

Dijagramom razreda prikazujemo razrede u sustavu, njihove atribute i metode te veze između razreda koji se nasljeđuju ili međusobno komuniciraju. U nastavku slijede dijagrami čiji razredi imaju sličnu funkcionalnost i razinu apstrakcije

EntityClassDiagram - prikazuje razrede koji pripadaju sloju Model. Svaki od razreda se preslikava u odgovarajuću tablicu u bazi

RepositoryClassDiagram – prikazuje razrede koji pripadaju sloju Repository. Ovaj sloj komunicira sa bazom podataka i sa slojem Service.

ServiceClassDiagram – prikazuje razrede koji pripadaju sloju Service. Ovaj sloj komunicira sa slojem Repository i Controller.

ControllerClassDiagram – prikazuje razrede sloja Controller. Ovaj sloj komunicira sa slojem Service i s frontend dijelom aplikacije.

EnumClassDiagram – prikazuje enumeracije koje koristi sloj Model.

ConfigClassDiagram – prikazuje razrede sloja Config. Ovaj sloj služi za sigurnosne potrebe i komuniciranje putem e-maila.

SecurityClassDiagram - prikazuje razrede sloja Security. Ovaj sloj služi za sigurnost i detalje o korisniku.

RequestClassDiagram – prikazuje razrede sloja Request. Ovaj sloj komunicira sa slojem Controller u svrhu validacije Post Request-a.

ResponseClassDiagram – prikazuje razrede sloja Response. Ovaj sloj komunicira sa slojem Controller u svrhu vraćanja Response poruke.

\begin{figure}[H]
	\includegraphics[scale=0.15]{dijagrami/Entity.png}
	\centering
	\caption{Dijagram razreda (Model)}
	\label{fig:EntityClassDiagram}
\end{figure}

\begin{figure}[H]
	\includegraphics[scale=0.22]{dijagrami/Repository.png}
	\centering
	\caption{Dijagram razreda (Repository)}
	\label{fig:RepositoryClassDiagram}
\end{figure}

\begin{figure}[H]
	\includegraphics[scale=0.1]{dijagrami/Service.png}
	\centering
	\caption{Dijagram razreda (Service)}
	\label{fig:ServiceClassDiagram}
\end{figure}

\begin{figure}[H]
	\includegraphics[scale=0.15]{dijagrami/Controller.png}
	\centering
	\caption{Dijagram razreda (Controller)}
	\label{fig:ControllerClassDiagram}
\end{figure}

\begin{figure}[H]
	\includegraphics[scale=0.2]{dijagrami/Enum.png}
	\centering
	\caption{Dijagram enumeracija (Model)}
	\label{fig:EnumClassDiagram}
\end{figure}

\begin{figure}[H]
	\includegraphics[scale=0.18]{dijagrami/Config.png}
	\centering
	\caption{Dijagram razreda (Config)}
	\label{fig:ConfigClassDiagram}
\end{figure}

\begin{figure}[H]
	\includegraphics[scale=0.18]{dijagrami/Security.png}
	\centering
	\caption{Dijagram razreda (Security)}
	\label{fig:SecurityClassDiagram}
\end{figure}

\begin{figure}[H]
	\includegraphics[scale=0.18]{dijagrami/Request.png}
	\centering
	\caption{Dijagram razreda (Request)}
	\label{fig:RequestClassDiagrams}
\end{figure}

\begin{figure}[H]
	\includegraphics[scale=0.16]{dijagrami/Response.png}
	\centering
	\caption{Dijagram razreda (Response)}
	\label{fig:ResponseClassDiagram}
\end{figure}



\eject

\section{Dijagram stanja}

Dijagram stanja prikazuje stanja objekta te prijelaze iz jednog stanja u drugo temeljene na dogadajima. Prilikom pokretanja aplikacije dolazi se na početnu stranicu (eng.Home page). Neregistrirani korisnik ima mogućcnost registracije i pregled kontakta i učestalih pitanja bez registracije

Registrirani korisnik na svojem početnom zaslonu nakon prijave može kliknuti na svoj profil gdje vidi svoje podatke, također na profilnoj stranici može deaktivirati račun i promjenit lozinku. Klikom na Dodaj novi termin korisnik se može prijaviti za termin koji može i naknadno otkazati.

Zaposlenik na svojem zaslonu može kao i korisnik kliknut na svoj profil gdje vidi svoje podatke i mjenja lozinku. Može davati termine i sobe pristiglim prijavama kao i pregledavati gotove termine te pisati rezultate terapije.

Superadmin ima uvid u statistiku, ima popis svih pacijenta i osoblja. Može dodavati, brisati osoblje te im dati uloge admina. Također ima popis svih soba i opreme koje može brisati i dodavati ili staviti da nisu u funkciji.

\begin{figure}[H]
	\includegraphics[scale=0.4]{dijagrami/bezprijave.png}
	\centering
	\caption{Dijagram stanja (neprijavljeni korisnik)}
	\label{fig:bezprijave}
\end{figure}

\begin{figure}[H]
	\includegraphics[scale=0.4]{dijagrami/korisnik.png}
	\centering
	\caption{Dijagram stanja (prijavljeni korisnik)}
	\label{fig:korisnik}
\end{figure}

\begin{figure}[H]
	\includegraphics[scale=0.4]{dijagrami/zaposlenik.png}
	\centering
	\caption{Dijagram stanja (zaposlenik)}
	\label{fig:zaposlenik}
\end{figure}

\begin{figure}[H]
	\includegraphics[scale=0.4]{dijagrami/superadmin.png}
	\centering
	\caption{Dijagram stanja (superadmin)}
	\label{fig:superadmin}
\end{figure}

\eject 

\section{Dijagram aktivnosti}

Pacijent započinje proces prijavom u sustav putem web-aplikacije, unoseći svoje korisničke podatke. Web-aplikacija šalje upit za provjeru tih podataka bazi podataka, koja zatim vraća rezultat. Ako su uneseni podaci neispravni, pacijent mora ponovno unijeti podatke, inače mu se prikazuje stranica za pacijente. Nakon uspješne prijave, pacijent odabire stvaranje nove terapije, što rezultira prikazom forme za unos podataka o bolesti i liječniku koji je dao uputnicu. Ponovno se šalje upit za provjeru tih podataka bazi, a ako su neispravni, pacijent ponovno unosi informacije; inače, liječnik prima zahtjev za terapijom i unosi podatke o datumu, vremenu i sobi za terapiju. Web-aplikacija ponovno šalje upit za provjeru podataka bazi, a u slučaju neispravnih podataka, liječnik ih ponovno unosi. U suprotnom, web-aplikacija šalje e-mail o potvrdi terapije i prikazuje stranicu za pacijenta s dodijeljenim terminom. Nakon toga, podaci o zahtjevu za terapijom spremaju se u sustav.

\begin{figure}[H]
	\includegraphics[scale=0.45]{dijagrami/Activity diagram.jpeg}
	\centering
	\caption{Dijagram aktivnosti}
	\label{fig:ActivityDiagram}
\end{figure}

\eject
\section{Dijagram komponenti}

Dijagram komponenti detaljno opisuje strukturu sustava, međusobne veze između komponenti te njihove odnose s okolinom. Prikazuje integraciju frontend i backend aplikacija putem REST API-ja. Backend aplikacija je složena od servisa, repozitorija i kontrolera. Repozitorij komunicira s bazom podataka putem SQL upita kako bi dohvatio potrebne podatke. Servis komunicira s repozitorijem i kontrolerom. Kontroler obrađuje dolazne zahtjeve s frontenda. Frontend aplikacija obuhvaća routere, komponente, servise te razne biblioteke. Router je komponenta koja, na temelju korisničkog zahtjeva za određeni URL, određuje koja datoteka (stranica) će biti isporučena. Komponente predstavljaju stranice koje se isporučuju kao JavaScript datoteke (HTML, CSS, JavaScript kod) ovisne o React biblioteci. Servisi imaju ulogu zahtjeva prema backendu, validacije i sličnih zadataka. Axios, kao biblioteka HTTP klijenta, pojednostavljuje slanje HTTP zahtjeva prema REST API-ju, odnosno prema backend aplikaciji.

\begin{figure}[H]
	\includegraphics[scale=0.17]{dijagrami/component_diagram.png}
	\centering
	\caption{Dijagram komponenti}
	\label{fig:component_diagram}
\end{figure}