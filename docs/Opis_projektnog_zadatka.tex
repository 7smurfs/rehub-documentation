\chapter{Opis projektnog zadatka „Medicinska rehabilitacija“}



\section{Uvod}

Cilj ovog projektnog zadatka je razviti programsku podršku za izradu web aplikacije „ReHub“ koja će olakšati prijavu i praćenje poboljšanja zdravstvenog stanja ljudi/bolesnika koji su imali lakše ili teže povrede nakon kojih je potrebno provesti fizikalnu terapiju i medicinsku rehabilitaciju. Razvijena kako bi omogućila transparentan, učinkovit, i siguran način za bolesnike da prate svoj put prema oporavku, „ReHub“ također pomaže zdravstvenim ustanovama bolje organizirati i pratiti rehabilitacijske procese.

\section{Izgled stranica aplikacije}

U nastavku se opisuju stranice web aplikacije te njihove funkcionalnosti.

\subsection{Početna stranica}

Na početnoj stranici korisnik može odabrati jednu od tri moguće opcije kako želi nastaviti rad u aplikaciji ili potražiti detaljnije informacije:

\begin{packed_item}
	
	\item  Prijava
	\item  Registracija
	\item  Korisnik
	
\end{packed_item}

\subsection{Stranica za prijavu}

Od korisnika se za uspješan nastavak rada u aplikaciji traži da unese valjano korisničko ime te ispravnu lozinku. Postoji i opcija „Zaboravljena lozinka“ kojom se korisniku na e-mail šalju upute za postavljanje nove lozinke u slučaju da je korisnik zaboravio lozinku.

\subsection{Zaboravljena lozinka}

Na ovoj stranici se od korisnika očekuje da unese e-mail adresu na koju želi dobiti daljnje upute o ponovnom postavljanju lozinke.

\subsection{Registracija}

Stranica sadrži polja u koje korisnik upisuje ime, prezime, e-mail adresu, broj telefona, OIB, lozinku, ponovno lozinku, MBO, datum rođenja te spol (opcionalno) ukoliko se želi uspješno registrirati.

\subsection{Pomoć}

Ova stranica sadrži odgovore na najčešće postavljena pitanja te sadrži detaljne upute kako se ispravno registrirati i prijaviti u aplikaciju.

\subsection{Kontakt}

Odabirom ove stranice korisnik može vidjeti e-mail adresu lokaciju te radno vrijeme organizacije.

\subsection{Početna stranica Admina}

Na početnoj stranici Admina mogu se vidjeti svi djelatnici i pacijenti u klinici. Gornji dio stranice sadrži opciju odjave te opciju biranja postavki.

\subsection{Početna stranica Korisnika}

Ovdje korisnik može vidjeti popis svojih termina te ima opciju kojom se naručuje na novi termin. Na vrhu stranice postoje opcije „Kontakt“, „Moj profil“ i „Odjava“.

\subsection{Naručivanje na termin}

Korisnik bira jednu od opcija „Prvi dolazak“ ili „Ponovna terapija“. Uz to navodi vrstu oboljenja, opis oboljenja, izdavatelja uputnice postupak liječenja i referencu na obavljenu terapiju.

\subsection{Početna stranica Djelatnika}

Djelatnik na svojoj početnoj stranici vidi popis prijava na termine, popis dostupnih i nedostupnih djelatnika, opciju za dodjeljivanje termina, popis slobodnih i nedostupnih soba i ordinacija te popis opreme.

\subsection{Dodjela termina}

Na ovoj stranici djelatnik ima pristup informacijama koje je korisnik naveo pri naručivanju termina. Djelatnik odabire ordinaciju, termin i djelatnika koji će raditi na rehabilitaciji.

\subsection{Stranica o terapiji}

Na ovoj stranici unose se podaci vezani uz terapiju.

\subsection{Korisničke postavke}

Na ovoj stranici korisnik ima pregled podataka i opcije postavki.

\section{Dodatno o aplikaciji}

Zamišljeno je da aplikacija bude što je više moguće lakša i intuitivnija za upotrebu, pošto računamo da bi pripadnici svih dobnih skupina mogli i htjeli koristiti ovu aplikaciju. U funkcionalnosti aplikacije vidimo veliku korist jer bi se njenom uporabom klijentima  znatno olakšalo i ubrzalo naručivanje na medicinsku rehabilitaciju, a djelatnicima klinike bi se olakšala evidencija naručenih klijenata. Klinike bi također mogle doživjeti povećani interes klijenata zainteresiranih za njihove usluge. Uporabom aplikacije klijenti bi izbjegli dugotrajna čekanja u redovima bolnica i klinika. Aplikacija osigurava sigurnost i zaštitu osjetljivih zdravstvenih podataka te poboljšava komunikaciju između bolesnika, zdravstvenih ustanova i liječnika. Smatramo kako je ključno pratiti i analizirati statistiku rehabilitacije radi kontinuiranog poboljšanja procesa. Aplikacija podržava tamnu temu.

\vspace{35pt}
\eject
\Large Ključne Funkcionalnosti \normalsize

\begin{packed_item}
	
	\item  \textbf{Registracija bolesnika:}\\
	Bolesnici se registriraju putem aplikacije, pružajući svoje osobne podatke.
	Administrator ustanove provjerava i verificira podatke za osiguranje točnosti informacija.
	
	\item  \textbf{Prijavljivanje za rehabilitaciju:}\\
	Bolesnici unose detalje o svojim oboljenjima i zahtjevima za liječenjem.
	Liječnici su verificirani putem imenika liječnika.
	
	\item  \textbf{Raspoređivanje termina:}\\
	Djelatnici ustanove pregledavaju i odobravaju prijave bolesnika.
	Na temelju kapaciteta opreme, prostora i osoblja, dodjeljuju termine za rehabilitaciju.
	
	\item  \textbf{Komunikacija s bolesnicima:}\\
	Bolesnicima se automatski šalju informacije o njihovim terminima putem elektroničke pošte.
	Mogućnost direktnog kontakta putem elektroničke pošte za obavještavanje o promjenama.
	
	\item  \textbf{Sigurnost podataka:}\\
	Visoka razina sigurnosti osigurava zaštitu osobnih podataka bolesnika i drugih osjetljivih informacija.
	
	\item  \textbf{Pregled i upisivanje rezultata:}\\
	Djelatnici ustanove imaju pristup svim podacima o bolesnicima i njihovim tretmanima.
	Nakon svakog rehabilitacijskog zahvata, upisuju postignute rezultate.
	
	\item  \textbf{Administrator:}\\
	Sistemski administrator ima potpunu kontrolu nad aplikacijom.
	Može definirati postavke, pristupne razine i konfiguracije sustava.
	
	
\end{packed_item}

\eject

